\documentclass[manuscript,screen,review]{acmart}


\IfFileExists{upquote.sty}{\usepackage{upquote}}{}
\IfFileExists{microtype.sty}{% use microtype if available
  \usepackage[]{microtype}
  \UseMicrotypeSet[protrusion]{basicmath} % disable protrusion for tt fonts
}{}
\makeatletter
\@ifundefined{KOMAClassName}{% if non-KOMA class
  \IfFileExists{parskip.sty}{%
    \usepackage{parskip}
  }{% else
    \setlength{\parindent}{0pt}
    \setlength{\parskip}{6pt plus 2pt minus 1pt}}
}{% if KOMA class
  \KOMAoptions{parskip=half}}
\makeatother

%%
%% This is file `sample-manuscript.tex',
%% generated with the docstrip utility.
%%
%% The original source files were:
%%
%% samples.dtx  (with options: `manuscript')
%% 
%% IMPORTANT NOTICE:
%% 
%% For the copyright see the source file.
%% 
%% Any modified versions of this file must be renamed
%% with new filenames distinct from sample-manuscript.tex.
%% 
%% For distribution of the original source see the terms
%% for copying and modification in the file samples.dtx.
%% 
%% This generated file may be distributed as long as the
%% original source files, as listed above, are part of the
%% same distribution. (The sources need not necessarily be
%% in the same archive or directory.)
%%
%%
%% Commands for TeXCount
%TC:macro \cite [option:text,text]
%TC:macro \citep [option:text,text]
%TC:macro \citet [option:text,text]
%TC:envir table 0 1
%TC:envir table* 0 1
%TC:envir tabular [ignore] word
%TC:envir displaymath 0 word
%TC:envir math 0 word
%TC:envir comment 0 0
%%
%%
%% The first command in your LaTeX source must be the \documentclass command.


% Options for packages loaded elsewhere
\PassOptionsToPackage{unicode}{hyperref}
\PassOptionsToPackage{hyphens}{url}
\PassOptionsToPackage{dvipsnames,svgnames,x11names}{xcolor}

\IfFileExists{bookmark.sty}{\usepackage{bookmark}}{\usepackage{hyperref}}

%% PANDOC PREAMBLE BEGINS


\providecommand{\tightlist}{%
  \setlength{\itemsep}{0pt}\setlength{\parskip}{0pt}}\usepackage{longtable,booktabs,array}
\usepackage{calc} % for calculating minipage widths
% Correct order of tables after \paragraph or \subparagraph
\usepackage{etoolbox}
\makeatletter
\patchcmd\longtable{\par}{\if@noskipsec\mbox{}\fi\par}{}{}
\makeatother
% Allow footnotes in longtable head/foot
\IfFileExists{footnotehyper.sty}{\usepackage{footnotehyper}}{\usepackage{footnote}}
\makesavenoteenv{longtable}
\usepackage{graphicx}
\makeatletter
\def\maxwidth{\ifdim\Gin@nat@width>\linewidth\linewidth\else\Gin@nat@width\fi}
\def\maxheight{\ifdim\Gin@nat@height>\textheight\textheight\else\Gin@nat@height\fi}
\makeatother
% Scale images if necessary, so that they will not overflow the page
% margins by default, and it is still possible to overwrite the defaults
% using explicit options in \includegraphics[width, height, ...]{}
\setkeys{Gin}{width=\maxwidth,height=\maxheight,keepaspectratio}
% Set default figure placement to htbp
\makeatletter
\def\fps@figure{htbp}
\makeatother

\definecolor{mypink}{RGB}{219, 48, 122}
\makeatletter
\@ifpackageloaded{caption}{}{\usepackage{caption}}
\AtBeginDocument{%
\ifdefined\contentsname
  \renewcommand*\contentsname{Table of contents}
\else
  \newcommand\contentsname{Table of contents}
\fi
\ifdefined\listfigurename
  \renewcommand*\listfigurename{List of Figures}
\else
  \newcommand\listfigurename{List of Figures}
\fi
\ifdefined\listtablename
  \renewcommand*\listtablename{List of Tables}
\else
  \newcommand\listtablename{List of Tables}
\fi
\ifdefined\figurename
  \renewcommand*\figurename{Figure}
\else
  \newcommand\figurename{Figure}
\fi
\ifdefined\tablename
  \renewcommand*\tablename{Table}
\else
  \newcommand\tablename{Table}
\fi
}
\@ifpackageloaded{float}{}{\usepackage{float}}
\floatstyle{ruled}
\@ifundefined{c@chapter}{\newfloat{codelisting}{h}{lop}}{\newfloat{codelisting}{h}{lop}[chapter]}
\floatname{codelisting}{Listing}
\newcommand*\listoflistings{\listof{codelisting}{List of Listings}}
\makeatother
\makeatletter
\makeatother
\makeatletter
\@ifpackageloaded{caption}{}{\usepackage{caption}}
\@ifpackageloaded{subcaption}{}{\usepackage{subcaption}}
\makeatother
%% PANDOC PREAMBLE ENDS

\setlength{\parindent}{10pt}
\setlength{\parskip}{0pt}

\hypersetup{
  pdftitle={Conceptualizing open health data platforms for low- and middle income countries},
  pdfauthor={Daniel Kapitan; Femke Heddema; Julie Fleischer; Chris Ihure; Antragama Abbas; Steven Wanyee; XXX; John Grimes; Mark van der Graaf; Mark de Reuver},
  pdfkeywords={template, demo},
  colorlinks=true,
  linkcolor={blue},
  filecolor={Maroon},
  citecolor={Blue},
  urlcolor={red},
  pdfcreator={LaTeX via pandoc, via quarto}}

%% \BibTeX command to typeset BibTeX logo in the docs
\AtBeginDocument{%
  \providecommand\BibTeX{{%
    Bib\TeX}}}

%% Rights management information.  This information is sent to you
%% when you complete the rights form.  These commands have SAMPLE
%% values in them; it is your responsibility as an author to replace
%% the commands and values with those provided to you when you
%% complete the rights form.
\setcopyright{}
\copyrightyear{}
\acmYear{}
\acmDOI{}

%% These commands are for a PROCEEDINGS abstract or paper.
\acmConference[]{}{}{}
\acmPrice{}
\acmISBN{}

%% Submission ID.
%% Use this when submitting an article to a sponsored event. You'll
%% receive a unique submission ID from the organizers
%% of the event, and this ID should be used as the parameter to this command.
%%\acmSubmissionID{123-A56-BU3}

%%
%% For managing citations, it is recommended to use bibliography
%% files in BibTeX format.
%%
%% You can then either use BibTeX with the ACM-Reference-Format style,
%% or BibLaTeX with the acmnumeric or acmauthoryear sytles, that include
%% support for advanced citation of software artefact from the
%% biblatex-software package, also separately available on CTAN.
%%
%% Look at the sample-*-biblatex.tex files for templates showcasing
%% the biblatex styles.
%%

%%
%% The majority of ACM publications use numbered citations and
%% references.  The command \citestyle{authoryear} switches to the
%% "author year" style.
%%
%% If you are preparing content for an event
%% sponsored by ACM SIGGRAPH, you must use the "author year" style of
%% citations and references.
%% Uncommenting
%% the next command will enable that style.
%%\citestyle{acmauthoryear}


%% end of the preamble, start of the body of the document source.
\begin{document}


%%
%% The "title" command has an optional parameter,
%% allowing the author to define a "short title" to be used in page headers.
\title{Conceptualizing open health data platforms for low- and middle
income countries}

%%
%% The "author" command and its associated commands are used to define
%% the authors and their affiliations.
%% Of note is the shared affiliation of the first two authors, and the
%% "authornote" and "authornotemark" commands
%% used to denote shared contribution to the research.


  \author{Daniel Kapitan}
  \orcid{0000-0001-8979-9194}
            \affiliation{%
                  \institution{PharmAccess Foundation}
                                  \city{Amsterdam}
                                  \country{the Netherlands}
                      }
          \affiliation{%
                  \institution{Eindhoven University of Technology}
                                  \city{Eindhoven}
                                  \country{the Netherlands}
                      }
        \author{Femke Heddema}
  
            \affiliation{%
                  \institution{PharmAccess Foundation}
                                  \city{Amsterdam}
                                  \country{the Netherlands}
                      }
        \author{Julie Fleischer}
  
            \affiliation{%
                  \institution{PharmAccess Foundation}
                                  \city{Amsterdam}
                                  \country{the Netherlands}
                      }
        \author{Chris Ihure}
  
            \affiliation{%
                  \institution{PharmAccess Kenya}
                                  \city{Nairobi}
                                  \country{Kenya}
                      }
        \author{Antragama Abbas}
  
            \affiliation{%
                  \institution{Delft University of Technology}
                                  \city{Delft}
                                  \country{the Netherlands}
                      }
        \author{Steven Wanyee}
  
            \affiliation{%
                  \institution{IntelliSOFT}
                                  \city{Nairobi}
                                  \country{Kenya}
                      }
        \author{XXX}
  
            \affiliation{%
                  \institution{ONA}
                                  \city{Nairobi}
                                  \country{Kenya}
                      }
        \author{John Grimes}
  
            \affiliation{%
                  \institution{Australian e-Health Research Centre}
                                  \city{Brisbane}
                                  \country{Australia}
                      }
        \author{Mark van der Graaf}
  
            \affiliation{%
                  \institution{PharmAccess Foundation}
                                  \city{Amsterdam}
                                  \country{the Netherlands}
                      }
        \author{Mark de Reuver}
  
            \affiliation{%
                  \institution{Delft University of Technology}
                                  \city{Delft}
                                  \country{the Netherlands}
                      }
      

%% By default, the full list of authors will be used in the page
%% headers. Often, this list is too long, and will overlap
%% other information printed in the page headers. This command allows
%% the author to define a more concise list
%% of authors' names for this purpose.
%\renewcommand{\shortauthors}{Trovato et al.}
%%  
%% The abstract is a short summary of the work to be presented in the
%% article.
\begin{abstract}
This document is only a demo explaining how to use the template.    
\end{abstract}

%%
%% The code below is generated by the tool at http://dl.acm.org/ccs.cfm.
%% Please copy and paste the code instead of the example below.
%%

%%
%% Keywords. The author(s) should pick words that accurately describe
%% the work being presented. Separate the keywords with commas.


%%
%% This command processes the author and affiliation and title
%% information and builds the first part of the formatted document.
\maketitle

\setlength{\parskip}{-0.1pt}

\subsection{Introduction}\label{sec-intro}

This is a dummy example only for the purpose to use this repo as a
template starter for creating new formats. For this
\texttt{article-format-template} we call our dummy article \texttt{aft}.

This quarto extension format supports PDF and HTML outputs.
\texttt{quarto-journals} is aiming at porting existing {\LaTeX} template
from journals to be used with quarto. PDF format is what require the
most work to fit the journals guideline, but Quarto offer a nice
rendering for HTML output too. This demo format template only use basic
HTML format without any customization for now.

\subsection{About Quarto Extensions formats And Quarto Journals
Article}\label{about-quarto-extensions-formats-and-quarto-journals-article}

First, please get familiar with the following resources:

\begin{itemize}
\tightlist
\item
  \href{https://quarto.org/docs/extensions/formats.html}{Creating
  Formats} in Quarto as part of the
  \href{https://quarto.org/docs/extensions/}{Extensions} mechanism.
\item
  \href{https://quarto.org/docs/journals/}{Journals Articles} for
  Quarto.
\end{itemize}

\subsection{Structure of this
repository}\label{structure-of-this-repository}

Everything for the extensions is in \texttt{\_extensions}. See Quarto
doc for details.

\begin{itemize}
\item
  In \texttt{partials}, you'll find the \texttt{.tex} partials that can
  be used and should be removed or tweaked,s
\item
  Your extension can make shortcodes and lua filters available. This
  document shows the effect of the one provided in the \texttt{aft}
  format.
\item
  \texttt{aft} format sets some defaults which are different from
  \texttt{pdf} or \texttt{html}, link setting links to URL in read
  inside PDF output.
\end{itemize}

Source repository for this template format can found on
\href{https://github.com/quarto-journals/article-format-template}{Github}

\subsubsection{\texorpdfstring{\texttt{\_extensions\textbackslash{}aft}}{\_extensions\textbackslash aft}}\label{extensionsaft}

In this folder you'll find everything that defines the extensions which
could be installed using \texttt{quarto\ install\ extension} or be part
of the template when using \texttt{quarto\ use\ template}

\begin{description}
\item[Format Metadata]
This is in \texttt{\_extension.yml} is where all the metadata about the
format are defined so that Quarto knows what to use. Adapt this file for
you own template.
\item[Partials]
In \texttt{partials}, there are the \texttt{.tex} files that will be
used as Pandoc's template. We provide here all the partials supported by
Quarto and custom one for this format. Quarto allows to provide partials
to ease the process of tweaking the default latex Pandoc's template and
keeping it up to date.\\
This template repo contains all the relevant partials that you can use
with Quarto \emph{as example}. We only tweaked \texttt{title.tex} to
show the usage of a custom partials called \texttt{\_custom.tex}.\\
\textbf{Only keep the partials that you need to tweak for the format you
are creating}

If you need to completely change the default template (i.g customizing
partials is not enough), then you need to provide your own template to
Pandoc based on
\href{https://github.com/quarto-dev/quarto-cli/blob/main/src/resources/formats/pdf/pandoc/template.tex}{\texttt{template.tex}}
and also using partials or not. This can be provided using the
\texttt{template} YAML key in \texttt{\_extension.yml} for Quarto to use
it.

This is considered advanced configuration as it will be harder to
maintain than only using partials but could be required for some
specific format. Be aware that this may lead to loose some Pandoc or
Quarto features tied to default \texttt{template.tex} content if you
remove some specific parts.
\item[Lua Filters]
Most of the time, custom formats will need Lua filters to provide
specific features like cross format supports or provides custom
shortcodes through the Quarto extension mechanism. Those filters will be
available to the user and could be used in the custom formats (according
to \texttt{\_extensions} metadata). We have provided two examples:

\begin{itemize}
\tightlist
\item
  \texttt{color-text.lua}, a Lua filter used to add color to inline text
  for PDF and HTML outputs using the same Markdown syntax
\item
  \texttt{shorcodes.lua}, a Lua filter which follow
  \href{https://quarto.org/docs/authoring/shortcodes.html\#custom-shortcodes}{Quarto
  custom shortcodes} guidelines to provide a \texttt{} shortcode to
  nicely print LaTeX in PDF and HTML.
\end{itemize}

\textbf{Remove or replace with your own Lua filters}
\item[Format resources]
Resources required by the format needs to be available. We have provided
two examples:

\begin{itemize}
\tightlist
\item
  \texttt{te.bst} is a biblio style file for demo. It has been
  downloaded from
  https://www.economics.utoronto.ca/osborne/latex/TEBST.HTM
  (http://econtheory.org/technical/te.bst) - Licence
  \href{https://www.latex-project.org/lppl/}{LPPL}
\item
  \texttt{aft.cls} is a dummy class file for this example format. It is
  a copy of official \texttt{article.cls}, the one provided in LaTeX
  installation (i.e at \texttt{kpsewhich\ article.cls}) and renamed as
  example (Licence LaTeX Project Public License)
\item
  \texttt{custom.scss} is a style file to have a custom theme for our
  HTML format so that our Lua filter feature \texttt{color-tex.lua}
  works.
\end{itemize}

Those files are referenced within the \texttt{\_extension.yml} to be
used with our example format.

\textbf{Remove and replace with your own resources}
\item[\texttt{.quartoignore}]
Sometimes it is useful to have some files only needed for reference or
for development. They should be available in the source repository but
not downloaded to the user when \texttt{quarto\ use\ template} is used.

\textbf{Use \texttt{.quartoignore} to register such file and folder (one
file or folder per line)}
\item[\texttt{style-guide} folder]
For \texttt{quarto-journals} format, use \texttt{style-guide} folder to
include any documentation and resourced used for format creation, like a
journal style guide or original \texttt{.tex} template. This folder is
already added in \texttt{.quartoignore} in this example repo.

\textbf{Remove, rename or add to this folder}
\item[\texttt{template.qmd}]
This file is the template document that shows how to use the custom
format. It will be downloaded with other resource by
\texttt{quarto\ use\ template}, and even offered to be renamed if the
name \texttt{template.qmd} is used.

This file will usually use the custom format (here \texttt{aft-pdf} and
\texttt{aft-html}) and show how to use the template. When you'll copy
this template, you should be able to render this document to the demo
format.

\textbf{Adapt this file to provide a suitable template for your custom
format}
\item[Other files]
Other files are needed by the template and are usually user provided -
they are not part of the custom format.

Here \texttt{bibliography.bib} is here to demo the usage of the bst file
from the custom format.

\textbf{Remove this file and provide a suitable one for your template}
\end{description}

\newpage{}

\subsection{Checklist: Creating a custom
format}\label{checklist-creating-a-custom-format}

Here is the checklist to help you know what to modify:

\begin{itemize}
\tightlist
\item
  Read the resources mentioned at the top,
\item
  Use this template repo to create a new repository for your format
  (Click on ``Use this template'' to create new github repo)
\item
  Once you are acquainted with the content, remove the resources that
  are there only as example (see above)
\item
  Update README by replacing \texttt{aft} and
  \texttt{Article\ Format\ Template} mentions for your journal format
\item
  Keep only the template partials that you need to tweak, and add custom
  ones if needed
\item
  Add any Lua filters for shortcodes and other that would be useful to
  create the expected output format
\item
  Add any external resource your format will need, and that should be
  part of the extension format that will be downloaded,
\item
  Check \texttt{\_extension.yml} is updated correctly
\item
  Modify the skeleton \texttt{template.qmd} to your format and add any
  required resources to be downloaded to user.
\item
  Check \texttt{.quartoignore} is updated which everything that should
  not be downloaded.
\item
  Publish a demo of you format to github pages of the repo by using
  \texttt{quarto\ publish} command
\end{itemize}

\subsection{Demo of some features found in this demo journal
template}\label{demo-of-some-features-found-in-this-demo-journal-template}

\subsubsection{Shortcode demo}\label{sec-shortcode}

PDF are rendered using {\LaTeX} but it is best if one can use a Markdown
syntax for cross format support.

\texttt{} used in source is a shortcode syntax where the shortcode is
included in the extension folder \texttt{\_extensions}

\subsubsection{Code chunk}\label{sec-chunks}

This format hide chunks by default as option has been set in
\texttt{\_extension.yml} file.

\subsubsection{Text color}\label{sec-summary}

Our format makes applying color on inline text possible using the
\texttt{{[}content{]}\{color=\textless{}name\textgreater{}\}} syntax.
Let's see an example.

Here we are using a special feature of our format which is the coloring
because \textcolor{mypink}{pink is a \textbf{nice} color}.

This is possible thanks to the Lua Filter included in the custom
extension format.

\subsubsection*{Using references}\label{using-references}
\addcontentsline{toc}{subsubsection}{Using references}

I did not read this book \citep{CameronTrivedi2013} but it must be
interesting.

Differences between \texttt{aft-html} and \texttt{aft-pdf}:

\begin{itemize}
\tightlist
\item
  For the HTML format, we are using Pandoc citeproc to include the
  bibliography. Here \texttt{reference-section-title} controls the title
  for the chapter that will be used.
\item
  For the PDF format, \texttt{natbib} is used by default and the
  bibliography is included with a title by the LaTeX template.
\end{itemize}

%% begin pandoc before-bib
%% end pandoc before-bib
%% begin pandoc biblio
%% end pandoc biblio
%% begin pandoc include-after
%% end pandoc include-after
%% begin pandoc after-body
%% end pandoc after-body

\end{document}
\endinput
%%
%% End of file `sample-manuscript.tex'.
